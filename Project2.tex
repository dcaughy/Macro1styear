\documentclass{article}
\usepackage{amsmath}
\usepackage{geometry}
\usepackage{hyperref}
\usepackage{graphicx}
\geometry{margin=1in}
\author{Derek Caughy}
\date{November 7, 2025}
\title{Project 2}

\begin{document}

\maketitle

Table \ref{agg} reports the overall change in welfare, consumption and output between the two economies. Notably, in the two economies the government budget was about 0.8695, with the reform economy having a consumption tax rate of about 49.23\%.  It's notable, that welfare increased alongside consumption, while the aggregate output decreased when switching to the reform economy. This can be explained by the elimination of the 30\% income tax increasing the relative returns to labour, while also increasing the returns to investment. That is, the removal of the income tax allows the household to both supply less labour and invest less in order to smooth consumption. This reduction in both inputs of production can be seen by the market clearing interest rate increasing dramitically between the two economies. It means that in equilibrium, households can invest and work less to afford more consumption than in the benchmark economy. 

	\begin{table}[ht]
	\centering
	\begin{table}[ht]
	\centering
	\begin{tabular}{|l|c|c|c|}
	\hline
	&\textbf{V}&\textbf{C}&\textbf{Y}\\\hline
	\textbf{Benchmark}&-5.1260&1.0432&2.9382\\\hline
	\textbf{Reform}&-4.6719&1.7662&2.3964\\\hline
	\textbf{Change}&0.4540&0.7230&-0.5418\\\hline
	\end{tabular}
	\caption{Aggregate Values of Household Value Function, Consumption, and Output across both economies.}
	\label{agg}
\end{table}

	\caption{Aggregate Values of Household Value Function, Consumption, and Output across both economies.}
	\label{agg}
\end{table}
\begin{table}[ht]
	\centering
	\begin{table}[ht]
\centering
	\begin{tabular}{|l|c|c|c|c|c|c|c|}
	\hline
	&$\theta_1$&$\theta_2$&$\theta_3$&$\theta_4$&$\theta_5$&$\theta_7$&$\theta_7$\\\hline
	\textbf{90+}&0.0705&0.2732&0.6557&0.8512&0.6363&0.2903&0.0741\\\hline
	\textbf{80-90}&0.0690&0.2733&0.6136&0.8632&0.7073&0.3034&0.0727\\\hline
	\textbf{60-80}&0.1255&0.5351&1.1899&1.6634&1.2944&0.5978&0.1534\\\hline
	\textbf{40-60}&0.1155&0.4718&1.1788&1.6220&1.2945&0.5972&0.1568\\\hline
	\textbf{20-40}&0.0928&0.4124&1.1026&1.5730&1.2912&0.5785&0.1462\\\hline
	\textbf{0-20}&0.0908&0.3740&0.9248&1.4055&1.1997&0.5585&0.1549\\\hline
	\end{tabular}
	\caption{Measure of $CE_1$ defined in paper across percentiles from stationary distribution reported in percent}
	\label{ce1}
\end{table}

	\caption{Consumption Equivalent Welfare Change across the percentiles of each labour shock state across percentiles of welath 		distribution. Reported in percent}
	\label{cbar}
\end{table}

Table \ref{cbar} reports the average consumption equivalent welfare change across the stationary distribution between the two economies. The average consumption equivalent welfare gain corresponding to $\overline{CE}_2$ is about 0.1319\%. Notably, all types of workers had some amount of positive welfare gain, however, we can see that this effect is most concentrated in the more moderate values of labour productivity, indexed by the middle columns. It's also true that these effects are more pronounced for households that are in the middle of the welath distribution as well. Inspecting the two stationary distributions reveals that the benchmark economy has a far more right-skewed distribution of welath values than the reform economy. Recall that the reform economy lowers investment due to the higher returns on investing.  The reform economy has a wealth distribution that is denser around the bottom percentiles, but only due to the fact that households need to save less to afford the same leisure consumption bundle as in the benchmark economy. This explains why the distribution of welfare gains is more dense at the bottom of the wealth values versus the top welath values. Households are overall better off at lower levels of wealth in the reform economy versus the benchmark. 






\end{document}