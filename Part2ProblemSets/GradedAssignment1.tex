\documentclass{article}
\usepackage{amsmath}
\usepackage{geometry}
\usepackage{amsfonts}
\geometry{margin=1in}
\usepackage{mathtools}
\usepackage{annotate-equations}


\author{Derek Caughy}
\title{ECO2100 Part 2 Homework 1}
\date{November 13, 2025}
\begin{document}
\maketitle
\section*{a)}
Social welfare is given by:\\
\\
\begin{equation*}
	S=\tikzmarknode{production}{f(k)}\tikzmarknode{employed}{(1-u)}-\tikzmarknode{costs}{vk}
\end{equation*}
\annotate[yshift=1em]{left}{production}{Production of firm in match}
\annotate[yshift=0.5em]{above}{employed}{Number of matches}
\annotate[yshift=-0.5em]{below}{costs}{total cost of vacancies}\\
\\
Note that we want to solve the steady state level of unemployment. First note that\\
\\
\begin{equation*}
	\tikzmarknode{utime}{\dot{u}}=\tikzmarknode{destruction}{\delta(1-u)}-\tikzmarknode{creation}{\frac{m(\lambda)}{\lambda}u}
\end{equation*}
\annotate[yshift=0.5em]{left}{utime}{change in unemployment across time, =0 in steady state}
\annotate[yshift=-1em]{below}{destruction}{rate of job destruction times employed}
\annotate[yshift=0.5em]{right}{creation}{mathcing rate of unemployed workers times unemployed}
\\
Thus in steady state:
\[0=\delta -(\delta+\frac{m(\lambda)}{\lambda})u\]
\[\iff u=\frac{\delta}{\delta+\frac{m(\lambda)}{\lambda}}=\frac{\lambda\delta}{\lambda\delta+m(\lambda)}\]
Furthermore, note that
\[\lambda=\frac{u}{v} \iff v=\frac{u}{\lambda}=\frac{\delta}{\lambda\delta+m(\lambda)}\]
Thus we can write the social planner's problem
\begin{align*}
	\max_{\lambda\geq0, k>0} &f(k)\frac{m(\lambda)}{\lambda\delta+m(\lambda)}-k\frac{\delta}{\lambda\delta+m(\lambda)}\\
\end{align*}
The Lagrangian can be written as:
\[\mathcal{L}(\lambda, k, \mu_1, \mu_2)=\frac{f(k)m(\lambda)-k\delta}{\lambda\delta+m(\lambda)}+\mu_1(k-\epsilon)+\mu_2\lambda\]
where $\epsilon>0$ is an arbitrary constant.

\section*{b)}
First note, that $\epsilon$ is arbitrary, therefore we can always choose some $\epsilon$ such that $k>\epsilon>0$, thus by complementary slackness $\mu_1=0$. Secondly, for finite 
The first order conditions are written as follows:
\begin{align*}
	\frac{f'(k)m(\lambda)-\delta}{\lambda\delta+m(\lambda)}&=0 \tag{k}\\
	\frac{f(k)m'(\lambda)\bigl(\lambda\delta+m(\lambda)\bigr)-\bigl(\delta+m'(\lambda)\bigr)\bigl(f(k)m(\lambda)-k\delta\bigr)}{\bigl(\lambda\delta+m(\lambda)\bigr)^2}+\mu_2&=0 \tag{$\lambda$}\\
\end{align*}
Inspecting condition (k) and taking the limit as $\lambda \rightarrow 0$ reveals that $\lambda\neq 0$. That is, 
\[\lim_{\lambda\rightarrow 0}\frac{f'(k)m(\lambda)-\delta}{\lambda\delta+m(\lambda)}=\frac{-\delta}{\infty}<0\]
By complementary slackness, it follows that $\mu_2=0$
Now inspect FOC ($\lambda$) we can derive:
\[f(k)m'(\lambda)\bigl(\lambda\delta+m(\lambda)\bigr)-\bigl(\delta+m'(\lambda)\bigr)\bigl(f(k)m(\lambda)-k\delta\bigr)=0\]
by the denominator necessairily being nonzero. This can be simplified:
\[\lambda f(k)m'(\lambda)-f(k)m(\lambda)+k\bigl(\delta+m'(\lambda)\bigr)=0\]
\[\iff \bigl(\lambda m'(\lambda)-m(\lambda)\bigr)f(k)+k\bigl(\delta+m'(\lambda)\big)=0\]
\[\iff \bigl(1-\frac{m(\lambda)}{\lambda m'(\lambda)}\bigr)f(k)+k\bigl(\frac{\delta}{\lambda m'(\lambda)}+\frac{1}{\lambda}\bigr)=0\]
Define
\[\eta(\lambda)=\frac{m(\lambda)}{\lambda m'(\lambda)}\]
Then re-write the above:
\[\bigl(1-\eta(\lambda)\bigr)+k\bigl(\delta \eta(\lambda) m(\lambda)+\bigr)\]






\section*{c)}
\begin{align}
	rJ(k)&=f(k)-w-\delta\bigl(J(k)-V(k)\bigr) \label{J} \tag{i}\\
	rV(k)&=-k+m(\lambda)\bigl(J(k)-V(k)\bigr) \label{V} \tag{ii} \\
	rW(k)&=w -\delta \bigl(W(k)-U(k)\bigr) \label{W} \tag{iii} \\
	rU(k)&=\frac{m(\lambda)}{\lambda}\bigl(W(k)-U(k)\bigr) \label{U} \tag{iv}
\end{align}

\section*{d)}

First, note that by $k=\bar{k}$ that the above Bellman equations are no longer functions of k. Furthermore, the equilibrium wage rate will be given by Nash Bargaining. That is 

\[w=\arg\max_\beta (W-U)^\beta(J-V)^{(1-\beta)}\]
\[=\arg\max_\beta \beta \ln(W-U)+(1-\beta)\ln(J-V)\]
The first order condition yields:
\[\beta\frac{\frac{\partial W}{\partial w}}{W-U}+(1-\beta)\frac{\frac{\partial J}{\partial w}}{J-V}=0\]
The partial derivatives come from the fact that W and J are functions of w, while V and U are not. Re-writing \ref{W}:
\[W=\frac{w-\delta U}{r+\delta}\]
Doing the same for J in terms of \ref{J} yields:
\[J=\frac{f(k)-w-\delta V}{r+\delta}\]
It becomes obvious that $\frac{\partial W}{\partial w}=-\frac{\partial J}{\partial w}$ thus the first order condition can be re-written:
\[\beta\frac{1}{W-U}-(1-\beta)\frac{1}{J-V}=0\]
\[\iff \beta(J-V)=(1-\beta)(W-U)\]
Now we note the free entry condition which means that $V=0$ we can re-write the FOC to be:
\begin{equation*}
	\eqnmarkbox[blue]{WU}{W-U}=\beta\bigl(W-U+\eqnmarkbox[red]{J2}{J}\bigr)
\end{equation*}
\annotate[yshift=-1em]{below}{WU}{$=\frac{w-rU}{r+\delta}$ from \ref{W}}
\annotate[yshift=-1em]{below}{J2}{$=\frac{f(k)-w}{r+\delta}$ from \ref{J}}\\
\\
\[\iff \frac{w-rU}{r+\delta}=\frac{\beta}{r+\delta}(f(k)-rU)\]
\begin{equation}
	\iff rU(\beta-1)=\beta f(k)-w \label{FOC} \tag{garbage}
\end{equation}
From the FOC we can also derive:
\[W-U=\frac{\beta}{1-\beta}J\]
From \ref{V} we can obtain
\[J=\frac{k}{m(\lambda)}\]
Thus:
\[W-U=\frac{\beta}{1-\beta}\frac{k}{m(\lambda)}\]
Which can be subsititued into \ref{U} to obtain
\[rU=\frac{m(\lambda)}{\lambda}\frac{\beta}{1-\beta}\frac{k}{m(\lambda)}=\frac{\beta}{1-\beta}\frac{k}{\lambda}\]
Substituing into \ref{FOC} and doing some simplifying:
\[-\beta \frac{k}{\lambda}=\beta f(k)-w\]
\[\iff w=\beta(f(k)+\frac{k}{\lambda})\]
Now inspect \ref{V} using J from \ref{J}
\[0=-k+m(\lambda)\frac{f(k)-w}{r+\delta}\]
\[\iff w=f(k)-\frac{r+\delta}{m(\lambda)}k\]
Thus
\[\beta(f(k)+\frac{k}{\lambda})=f(k)-\frac{r+\delta}{m(\lambda)}k\]
\[\iff (1-\beta) f(k)-k\bigl(\frac{\beta}{\lambda}+\frac{r+\delta}{m(\lambda)}\bigr)=0\]








\end{document}