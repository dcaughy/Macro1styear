\documentclass{article}
\usepackage{amsmath}
\usepackage{geometry}
\usepackage{amsfonts}
\geometry{margin=1in}
\usepackage{mathtools}
\usepackage{annotate-equations}


\author{Derek Caughy}
\title{ECO2100 Part 2 Homework 1}
\date{November 13, 2025}
\begin{document}
\maketitle
\section*{a)}
Social welfare is given by:\\
\\
\begin{equation*}
	S=\tikzmarknode{production}{f(k)}\tikzmarknode{employed}{(1-u)}-\tikzmarknode{costs}{vk}
\end{equation*}
\annotate[yshift=1em]{left}{production}{Production of firm in match}
\annotate[yshift=0.5em]{above}{employed}{Number of matches}
\annotate[yshift=-0.5em]{below}{costs}{total cost of vacancies}\\
\\
Note that we want to solve the steady state level of unemployment. First note that\\
\\
\begin{equation*}
	\tikzmarknode{utime}{\dot{u}}=\tikzmarknode{destruction}{\delta(1-u)}-\tikzmarknode{creation}{\frac{m(\lambda)}{\lambda}u}
\end{equation*}
\annotate[yshift=0.5em]{left}{utime}{change in unemployment across time, =0 in steady state}
\annotate[yshift=-1em]{below}{destruction}{rate of job destruction times employed}
\annotate[yshift=0.5em]{right}{creation}{mathcing rate of unemployed workers times unemployed}
\\
Thus in steady state:
\[0=\delta -(\delta+\frac{m(\lambda)}{\lambda})u\]
\[\iff u=\frac{\delta}{\delta+\frac{m(\lambda)}{\lambda}}=\frac{\lambda\delta}{\lambda\delta+m(\lambda)}\]
Furthermore, note that
\[\lambda=\frac{u}{v} \iff v=\frac{u}{\lambda}=\frac{\delta}{\lambda\delta+m(\lambda)}\]
Thus we can write the social planner's problem
\begin{align*}
	\max_{\lambda\geq0, k>0} &f(k)\frac{m(\lambda)}{\lambda\delta+m(\lambda)}-k\frac{\delta}{\lambda\delta+m(\lambda)}\\
\end{align*}
The Lagrangian can be written as:
\[\mathcal{L}(\lambda, k, \mu_1, \mu_2)=\frac{f(k)m(\lambda)-k\delta}{\lambda\delta+m(\lambda)}+\mu_1(k-\epsilon)+\mu_2\lambda\]
where $\epsilon>0$ is an arbitrary constant.

\section*{b)}
First note, that $\epsilon$ is arbitrary, therefore we can always choose some $\epsilon$ such that $k>\epsilon>0$, thus by complementary slackness $\mu_1=0$. Secondly, for finite 
The first order conditions are written as follows:
\begin{align*}
	\frac{f'(k)m(\lambda)-\delta}{\lambda\delta+m(\lambda)}&=0 \tag{k}\\
	\frac{f(k)m'(\lambda)\bigl(\lambda\delta+m(\lambda)\bigr)-\bigl(\delta+m'(\lambda)\bigr)\bigl(f(k)m(\lambda)-k\delta\bigr)}{\bigl(\lambda\delta+m(\lambda)\bigr)^2}+\mu_2&=0 \tag{$\lambda$}\\
\end{align*}
Inspecting condition (k) and taking the limit as $\lambda \rightarrow 0$ reveals that $\lambda\neq 0$. That is, 
\[\lim_{\lambda\rightarrow 0}\frac{f'(k)m(\lambda)-\delta}{\lambda\delta+m(\lambda)}=\frac{-\delta}{\infty}<0\]
By complementary slackness, it follows that $\mu_2=0$
Now inspect FOC ($\lambda$) we can derive:
\[f(k)m'(\lambda)\bigl(\lambda\delta+m(\lambda)\bigr)-\bigl(\delta+m'(\lambda)\bigr)\bigl(f(k)m(\lambda)-k\delta\bigr)=0\]
by the denominator necessairily being nonzero. This can be simplified:
\[\lambda f(k)m'(\lambda)-f(k)m(\lambda)+k\bigl(\delta+m'(\lambda)\bigr)=0\]


\section*{c)}
\begin{align}
	rJ(k)&=f(k)-w-\delta\bigl(J(k)-V(k)\bigr) \label{J} \tag{i}\\
	rV(k)&=-k+m(\lambda)\bigl(J(k)-V(k)\bigr) \label{V} \tag{ii} \\
	rW(k)&=w -\delta \bigl(W(k)-U(k)\bigr) \label{W} \tag{iii} \\
	rU(k)&=\frac{m(\lambda)}{\lambda}\bigl(W(k)-U(k)\bigr) \label{U} \tag{iv}
\end{align}

\section*{d)}

First, note that by $k=\bar{k}$ that the above Bellman equations are no longer functions of k. Furthermore, the equilibrium wage rate will be given by Nash Bargaining. That is 

\[w=\arg\max_\beta (W-U)^\beta(J-V)^{(1\beta)}\]
\[=\arg\max_\beta \beta \ln(W-U)+(1-\beta)\ln(J-V)\]




\end{document}